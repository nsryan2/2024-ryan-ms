%% Create an abstract that can also be used for the ProQuest abstract.
%% Note that ProQuest truncates their abstracts at 350 words.
% This is the abstract.


% The nuclear fuel cycle plays a critical role in designing sustainable and efficient nuclear energy systems. The goal of this research is to expand nuclear fuel cycle modeling by removing simplifying assumptions and improving the accuracy of simulations used for reactor and fuel cycle design. This work uses the open-source Cyclus code to model fuel cycles for the AP1000, X-Energy Xe-100, and USNC MMR. We also improve the precision of reactor core loading, fuel supply chain dynamics, and transaction modeling. We implement new reactor models called the Dynamic Power Reactor (DPR), Trading On-Demand (TOD) reactor, and the Enrichment Versatile Reactor (EVER) to allow flexible fuel usage. Additionally, we examine transition scenarios where LEU+ fuel delays the demand for HALEU. By evaluating these factors, the research contributes to more efficient fuel cycle design, with an emphasis on improving existing tools for advanced reactor fuel cycle simulations. The results show that improving reactor models and simulating fuel cycle transitions leads to more efficient reactor deployment and fuel cycle design. This work contributes to the development of advanced reactor fuel cycle simulations that provide better flexibility and accuracy in real-world energy planning. Future work will refine these models, expanding their use in real-world energy planning and reactor optimization.

% Context:
% The nuclear fuel cycle plays a critical role in designing sustainable and efficient nuclear energy systems. This research focuses on enhancing fuel cycle modeling to improve reactor and fuel cycle design accuracy, with particular attention to advanced reactor technologies.

% Questions:
% This study aims to address the need for more accurate simulations of fuel cycles for advanced reactors, including the AP1000, X-Energy Xe-100, and USNC Micro Modular Reactors (MMRs). It investigates how to improve reactor core loading, fuel supply chain dynamics, and transaction modeling, while incorporating new reactor models and transition scenarios.

% Hypothesis/Prediction:
% By removing simplifying assumptions and improving simulation precision, the study hypothesizes that more accurate fuel cycle models will lead to improved reactor deployment strategies, better fuel utilization, and a more efficient nuclear fuel cycle design.

% Methods:
% Using the open-source Cyclus code, this research models fuel cycles for the AP1000, Xe-100, and MMR reactors. New reactor models—Dynamic Power Reactor (DPR), Trading On-Demand (TOD) reactor, and Enrichment Versatile Reactor (EVER)—are developed to allow flexible fuel usage. Transition scenarios, including those involving LEU+ fuel, are evaluated to assess their impact on the demand for HALEU.

% Results:
% The results show that improving reactor models and simulating fuel cycle transitions leads to more efficient reactor deployment and fuel cycle design. This work contributes to the development of advanced reactor fuel cycle simulations that provide better flexibility and accuracy in real-world energy planning.

% Conclusion/Significance:
% This research enhances existing tools for advanced reactor fuel cycle modeling, supporting more efficient reactor optimization and contributing to future nuclear energy solutions. Future work will refine these models, further expanding their use in energy planning and reactor design.


\textbf{Keywords:} Cyclus, TRISO, HALEU, LEU+, LEU Plus, Serpent, Nuclear Fuel Cycle, Memory Efficiency