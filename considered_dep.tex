\chapter{Considered Deployment Schemes}
\label{sec:considered_deployment_schemes}
In addition to the deployment schemes outlined in \ref{sec:greedy_deployment},
\ref{sec:random_deployment}, and \ref{sec:initially_random_greedy}, we also
considered a few others that we did not include in the final analysis. We
examined these schemes for their potential to capture the complexity of the
deployment problem but were ultimately not included due to the egregious nature
of approximation required to implement or the lack of a clear benefit over the
other schemes for the questions explored in this work.

\section{Capped Deployment}
\label{sec:capped_deployment}
This scheme places a constant limit on the number of specific reactors deployed
at any given time step. This is a simple way to model aggregate supply chain
constraints that could limit vendors from deploying reactors freely. With the
right constraints, this scheme would better succeed at roughly incorporating
the limits of a workforce over a short to medium time scale. As workforce
constraints are outside the scope of this work, we implement this scheme but do
not incorporate it into this work.

To use this deployment scheme, a user needs to understand the supply chain
constraints that will limit the deployment of the reactors they are deploying.
We illustrate the defining steps of the capped deployment scheme in Figure
\ref{fig:cap_diagram}. The main loop in the logic is consistent with the greedy
deployment scheme but adds a check to see if the current deployment exceeds the
limit on that reactor. If it does, the reactor is removed from the list of
reactors to be deployed in that time step.

\begin{figure}[H]
    \centering
    \includegraphics[scale=0.4]{images/schemes/cap_diagram.png}
    \caption{Capped deployment diagram.}
    \label{fig:cap_diagram}
\end{figure}

The realism of this deployment scheme mirrors some elements of the
pre-determined distribution (this is a flat distribution after all), but the
cap is a less granular way to account for supply chain constraints. This scheme
is most useful for scenarios or timescales where there is a known limit on the
workforce. The unrealistic element of this deployment scheme comes from two
places: 1) the current implementation requires one reactor to be unrestrained
(preferably the smallest reactor from the deployment standpoint); 2) the cap is
a flat distribution, which is not a realistic representation of the supply
chain constraints for most technologies.

When the unconstrained reactor is not the smallest power reactor, this scheme
will fall below demand moreso than when the unconstrained reactor is the
smallest in power. This scheme has the potential to overperform by one reactor
in the case where the unconstrained reactor is the smallest in power as it can
over-deploy by one reactor's capacity in that case.

\section{Pre-Determined Distribution Deployment}
\label{sec:pre_determined_distribution_deployment}
This deployment scheme allows users to incorporate the projections and
commitments of ratepayers and utilities by setting a distribution over the
simulation time. In this scheme, the distribution serves as a cap to the
number of reactors deployed in a time step, and we preferentially
deploy reactors first to meet those caps. After completion, we deploy the
remaining reactors without caps to meet the demand. In this way, we incorporate
knowledge of supply chain constraints for specific technologies without having
to model the supply chain in detail.

To use this deployment scheme, a user needs some idea of the distribution of
reactors deployed over the simulation time. We illustrate the defining steps of
the pre-determined distribution deployment scheme in Figure
\ref{fig:pre_det_diagram}. The main loop in the logic is consistent with the
greedy deployment scheme but adds a check to see if the current deployment
exceeds the limit on that reactor. If it does, the scheme removes the reactor
from the list of deployable reactors in that time step. This scheme varies from
the capped deployment scheme in that the distribution is not flat, but a more
granular distribution that varies by year.

\begin{figure}[H]
    \centering
    \includegraphics[scale=0.4]{images/schemes/pre_det_diagram.png}
    \caption{Pre-determined distribution deployment diagram.}
    \label{fig:pre_det_diagram}
\end{figure}

The realism of this deployment scheme mirrors some elements of the capped
deployment, but the distribution is a more granular way to account for supply
chain constraints. This scheme is most useful when there are known commitments
to specific technologies. It allows the user to indirectly incorporate the
evolution of supply chains or workforce constraints over time, and to
explicitly incorporate decisions from individual actors. If a user established
the nuances of the supply chain constraints in other work, it could be
incorporated through this scheme. Under and over-performance of this scheme is
difficult to predict, as it depends on the distribution of reactors over time.