\section{Dynamic Power Reactor}
\label{sec:dpr_method}

The \gls{nrc} publishes a daily Power Reactor Status report for each reactor
under its jurisdiction \cite{nrc_power_2025}. These reports contain, amongst
other things, the percentage of the total power at which the operators
say the reactor operated. In the case of a fuel cycle simulation containing a
small number of reactors or a full-fleet simulation over a short time, the
differences in the power predicted by the \cycamore reactor and reality can
diverge.

Figure \ref{fig:pp_full} examines the single reactor operating at the
\gls{clinton}, with a reference unit power (i.e., net power) of 1062
MWe according to the \gls{iaea} \gls{pris} database \cite{IAEA_PRIS}, and
compare it to the results from the \cycamore reactor modeled over the same time
frame. This figure excludes the startup of the \cycamore reactor to ensure that it was operating on the same schedule as the data from the \gls{nrc} suggest the reactor was operating on from the start of 2021 through the end of 2024.

\begin{figure}[H]
  \centering
  \includegraphics[width=0.7\linewidth]{images/power_reactor/power_percent_clinton_fake.pdf}
  \caption{\gls{clinton} reactor daily capacity factor 2021-2024.}
  \label{fig:pp_full}
\end{figure}

A simple numerical integration reveals that the total energy capacity of both reactors differs by just under 51 GWe with a percent difference of 3.52\%. This difference can be negated by comparing it to a base case, but for the small-scale model, users might be interested in incorporating realistic fluctuations in power and find that the two scenarios in Figure \ref{fig:pp_full} were not equal on 908 days, or 62.2\%, of the 1460-day simulation. This thesis introduces the \gls{dpr} to mirror this variability in power of an operating reactor, as shown in Figure \ref{fig:pp_full}. \gls{dpr} functions the same way as the \cycamore reactor, except the user can input the percentage of the total capacity the reactor is outputting at any given time step.


\begin{figure}[H]
  \centering
  \includegraphics[width=0.7\linewidth]{images/power_reactor/dpr_diff.pdf}
  \caption{Difference in daily capacity factor between \gls{dpr} and \gls{clinton}.}
  \label{fig:dpr_clinton_diff}
\end{figure}


Narrowing the scope of this study to 2024, Figure \ref{fig:dpr_clinton_diff} shows how \gls{dpr} replicates capacity factor fluctuations with the difference between the reported values from the \gls{nrc} \cite{nrc_power_2025} and the results from the \cyclus simulation. The maximum difference between the two is $2.22 \times 10^{-16}$, which is explainable by floating point error in calculations as this value matches a double point machine epsilon value. Figure \ref{fig:dpr_cycamore_power} compares \gls{dpr} to the \cycamore reactor. As the reactors are assumed to start operations before 2024, a buffer month in which the reactors receive fuel allows these results to align with reality. The vertical line indicates when 2024 begins, allowing Figure \ref{fig:dpr_cycamore_power} to compare the \gls{nrc} data with results from \cyclus.


\begin{figure}[H]
  \centering
  \includegraphics[width=0.7\linewidth]{images/power_reactor/dpr_cycamore_energy.pdf}
  \caption{2024 capacity factor of the \cycamore reactor and \gls{dpr}.}
  \label{fig:dpr_cycamore_power}
\end{figure}


