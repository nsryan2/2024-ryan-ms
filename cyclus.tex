\pagebreak
(((((((((((((remove page break)))))))))))))
\section{\cyclus}
% fix that, too informal and introduces archetypes without context.
\cyclus is an agent-based \gls{nfc} simulator that is incredibly
versatile, one of the initial, and core, developers (Professor Katy Huff)
likes to say that, "\cyclus can be used to model any process from making
a grilled-cheese to international nuclear fuel cycles." The software
achieves this versatility through a series of generic archetypes that are
primarily transaction based. Over the years, the user community and
developers have created a litany of nuclear specific archetypes for
everything from proliferation assessment to fuel burnup.

% discuss archetypes
Users can incorporate different fuel cycle facilities into their \cyclus
simulations using generic facilities that are referred to as archetypes
in the \cyclus ecosystem. Many standard fuel cycle facilities hae been
implemented in the \cycamore repository, which holds technology agnostic
archetypes for material sources, material sinks, enrichment services,
separations capabilities, and a generic reactor.

% discuss recipes
As \cyclus is a transactions code and not necessarily a physics code,
the reactors incorporate reactor physics through pre-defined "recipes,"
where the user specifies the isotopic concentration of the fresh and
used fuel.

Users approximate the burnup of each fuel element with the
same input recipe as the same; however, in this work we incorporate a
cascading enrichment from \gls{leu+} to \gls{haleu}.
% find a citation or source that companies are actually going to do that
% (best case scenario is find it for each reactor you do it for)

% discuss EVER and CLOVER?
Novel in this work is our use of a low fidelity archetype based on the
\cycamore reactor %\cite{the summer poster}. \gls{ever}

\gls{ever} allows the user to specify multiple recipes for the fuel and
change between them at specific times.


% discuss DRE
As we have discussed, \cyclus's primary function is to keep track of
material transactions between agents. This is accomplished through the
\gls{dre}, which functions like a market where each agent brings a bid
for what and how much material they need and suppliers are matched with
buyers % cite something here.

% discuss time step stuff?
In this work we have incorporated and analyzed an alteration to the
frequency that each agent interacts with the \gls{dre} to better
understand the potential for simulation efficiency in run time and
memory usage.