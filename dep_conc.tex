\section{Transition Scenarios Conclusions}
\label{sec:dep_conc}

% 1. What are the SWU requirements for TRISO fueled reactor transitions that incorporate the proposed LEU+ to HALEU scheme?
% 2. If we have x growth of demand met by ARs, how much TRISO do we need, and when do we meet the demand?
% 3. What are the impacts of deployment schemes in NFC scenarios, and what parts are realistic or unrealistic in each? How quickly/often does the scenario meet the energy demand (we would need to identify limiting factors)?

This thesis characterizes bounding scenarios of reactor deployment as the \gls{usa} transitions to a new fleet of nuclear reactors using an interstitial fuel cycle involving \gls{leup} \gls{triso} fuel. The results of this thesis show that the reactor deployment scheme has an understandable and identifiable impact on the \gls{swu} required to meet the energy demand.

The transition scenarios from this thesis propose a series of fuel cycles that include and contextualize the deployment of new nuclear reactors in the \gls{usa} using \gls{triso} fuels at \gls{haleu} and \gls{leup} enrichment levels. These fuel cycles meet the energy demand growths predicted by the \gls{eia} and the \gls{doe}. The increase in demand assumes that nuclear energy's generation share remains constant over time. A 15$\%$ increase in demand means a 15$\%$ increase in nuclear energy capacity (as the EIA numbers \cite{eia_aeo_2023} reflect the total energy demand, this conversion is only possible by assuming that the percentage share of nuclear capacity is the same). This assumption is not reflected in the demand scenarios from the \gls{doe} liftoff report \cite{julie_liftoff_pathways_2024}, which are specific to nuclear deployment increases, and the number is agnostic to the total increase. The liftoff report scenarios are assumed to continue beyond the initial 2050 projection.

The \gls{lwr} fleet deviates from reality by assuming that they all have regular 18-month cycles with regular 1-month outages. Related to this consistent cycle length, \cyclus also assumes that the reactors have constant power outputs (when not in an outage) over their lifetime. Each \gls{lwr} has an assembly size of 427.38589211618256 and a batch size of 80, further normalizing the fleet. After 2024, no new \glspl{lwr} will be built other than the AP1000s deployed under the various schemes.

The supply chain is not a limiting factor for new reactors, which allows these simulations to characterize the upper bound of what needs to be in place to achieve the projected deployment scenarios. Additionally, the fabrication and enrichment facilities for fuel are a black box, not factoring in variations in time, resources, or regulations associated with the fuels included.

The \gls{mmr} and \gls{xe} Serpent simulations are based on limited, publicly available information and do not rely on confidential or proprietary data, another limitation is they assume that when the reactors accept \gls{leup} fuel and operate with the same power levels and burnup rates. As Section \ref{sec:reactor_models} discusses, the intended use of many advanced reactors extends beyond simply meeting energy demand, so modeling them entirely in on-the-grid applications is not necessarily the only way they will deploy. These reactors could provide a range of services that can contribute to decarbonizing the economy.

Table \ref{tab:swu_incs} shows the average yearly percentage \gls{swu} increases from the \textit{no growth} scenario to the \textit{double by 2050} scenario, and the impacts of the deployment scheme are reflected in these values. Where the greedy scheme regularly prefers the highest capacity reactors, leading to the most significant increase in the \gls{swu} for AP1000 \gls{leu}, the randomness in the other two schemes levels the reactor preferences and leads to more consistent increases across fuel types.

\begin{table}[H]
    \centering
    \caption{Average yearly percentage \gls{swu} increases from the no growth scenario to the double by 2050 scenario.}
    \label{tab:swu_incs}
    \begin{tabular}{l l l l}
        \hline
        Scheme & \gls{mmr} \gls{haleu} & \gls{xe} \gls{haleu} & AP1000 \gls{leu}\\
        \hline
        Greedy Deployment & 105\% & 167\% & 800\% \\
        Random Deployment & 1511\% & 796\% & 697\% \\
        Initially Random Greedy & 775\% & 672\% & 696\% \\
        \hline
    \end{tabular}
\end{table}

% The mass of fresh fuel required to deploy the reactors as proposed in this thesis is staggering. As shown in Table \ref{tab:fresh_incs}, the percentage increase of average fresh fuel mass from the \textit{no growth} scenario to the double scenario mirrors the values from Table \ref{tab:swu_incs}.

% \begin{table}[H]
%     \centering
%     \caption{Average yearly percentage fresh fuel increases from the \textit{no growth} scenario to the double scenario.}
%     \label{tab:fresh_incs}
%     \begin{tabular}{c c c c}
%         \hline
%         Scheme & \gls{mmr} \gls{haleu} & \gls{xe} \gls{haleu} & AP1000 \gls{leu}\\
%         \hline
%         Greedy Deployment & 105\% & 159\% & 800\% \\
%         Random Deployment & 1505\% & 796\% & 696\% \\
%         Initially Random Greedy & 775\% & 671\% & 696\% \\
%         \hline
%     \end{tabular}
% \end{table}



\subsection{Future Work}
\label{sec:future_work}

This thesis could be expanded in various ways to contextualize the deployment of advanced reactors in the \gls{usa}. Outside of simply removing the assumptions outlined above, two immediate additions to this thesis would be to incorporate isotope calculations for used fuel to understand the accumulation of isotopes of interest. This would allow for a more detailed understanding of the waste stream and the potential for recycling. The second addition would be to compare the \textit{no growth} and \textit{double by 2050} nuclear scenarios to the triple nuclear scenario from the \gls{doe} liftoff report \cite{julie_liftoff_pathways_2024}. These two additions were not included due to computational limitations and the sheer size of the data needed.

The next steps in expanding this thesis would be to translate the base metrics presented here (\gls{swu}, mass, energy, and deployment) into costs for fuel and energy. The mass of used fuel would be a good starting point for repository space considerations and transportation costs. This thesis could leverage \cyclus's ability to track latitude and longitude to understand the transportation time between facilities.

Section \ref{sec:leup} on \gls{leup} highlights how the category of an enrichment facility is a critical component in the cost and logistics of a fuel cycle. \gls{swu} is a good starting point for understanding the relative effort required to deploy the reactors, but the cost of that effort is a critical component of the deployment for making policy recommendations. \gls{swu} and masses of fuel can start to understand how international cooperation with nations that have existing enrichment facilities could help the \gls{usa} meet its energy goals.

% assumptions
% \begin{itemize}
    % \item The demand increase assumes that nuclear energy's share of generation remains constant over time. When we say a 15$\%$ increase in demand, we mean a 15$\%$ increase in nuclear energy generation (as the EIA numbers \cite{eia_aeo_2023} are meant to reflect the total energy demand, this conversion is only possible by assuming that the percentage share of nuclear capacity is the same). This assumption is not reflected in the demand scenarios from the \gls{doe} liftoff report \cite{julie_liftoff_pathways_2024}, which are specific to nuclear deployment increases and the number is agnostic to the total increase. The liftoff report scenarios are assumed to continue beyond the initial 2050 projection.
    % \item \gls{lwr}s have an 18 month cycle with no deviation.
    % \item \gls{lwr}s have a regular 1 month outage.
    % \item \gls{lwr}s have an assembly size of 427.38589211618256.
    % \item \gls{lwr}s have a batch size of 80.
    % \item All reactors have constant power output (when not in outage) over their lifetime.
    % \item No new \gls{lwr}s are built after 2024. % not really the case anymore
    % \item \gls{lwr}s have no additional outages other than the regular 1 month outage.
    % \item Essentially assume that the supply chain is not a limiting factor in the deployment of new reactors, we make no requirements that it is necessarily located in the \gls{usa}, but we make no effort to explicitly realize the global nature of the supply chain.
    % \item We treat the fabrication and enrichment of fuel as a black box, and do not consider the variance in time/ resources/ regulation required for the fuels we include.
    % \item MMR and \gls{xe} Serpent models are based on small amounts of publicly available information and are not based on confidential or proprietary information. They also assume that the \gls{haleu} fueled \gls{triso} reactors will first accept \gls{leup} fuel and operate at the same power level and similar burnup.
    % \item We assume that reactors will transition from \gls{leup} to \gls{haleu} fuel when it becomes available at the same time in 2040.
%     \item
% \end{itemize}


% limitations
% \begin{itemize}
%     \item The \gls{eia} is taking a year off in producing their report to better account for increased behind-the-meter investment, AI needs, and data center expansions \cite{eia_annual_outlook_canceled_2023}; they have indicated that these factors could be substantial.
%     \item Do not directly incorporate international understanding of the supply chain.
% \end{itemize}