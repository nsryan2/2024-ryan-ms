\section{Energy System Modeling}
\label{sec:esm}

Nations such as the \gls{usa} and the \gls{uk} centralized their electrical infrastructure as it developed, stemming from an attitude that Dieter
Helm from the University of Oxford describes as prevailing until the end
of the 1970s \cite{helm_energy_2002}. Due to the heavy state involvement,
energy planning is a concept that has existed for many decades. Evidence of contemporary planning can be found as far back as 1967 in
nationalized industry reports from the \gls{uk}
\cite{treasury_nationalised_1967} and was a top-of-mind consideration in the
\gls{usa} and other countries.

In 1973, Michael Posner from the University of Cambridge published his book
\textit{Fuel Policy A Study In Applied Economics} \cite{posner_fuel_1973},
which describes methods large institutions could use to make
energy decisions. In connection with the 1973 oil crisis, this book was a wake-up call for many countries to enhance their predictive capabilities for energy markets. The crisis led to the development of energy planning models
that could be used to evaluate the impact of different policies on energy
systems as disruptions tend to do \cite{plazas_disrupt_2022}. The \gls{iiasa},
founded in 1972, and the \gls{iea}, founded in 1974, have served international communities with \gls{esm} tools since the oil crisis. The resulting models were used to develop long-term energy plans to help countries increase their energy security, facilitate economic development, and better legislate with increasingly complex energy systems.

Today, utilities, countries, and other organizations use \glspl{esm} to model
the behavior of energy systems in different economic contexts, such as the cost
of energy, the price of carbon, and the availability of financing. These
contexts can focus on developing favorable conditions for new technologies,
understanding the relationship between actors, predicting future trends, and
the impact of different policies on energy systems. Decision-makers compare the
behavior of energy systems in various scenarios to a baseline, such as
business-as-usual scenarios compared with low-carbon or high-renewable
scenarios. These are effective across regulated, competitive, and hybrid
markets. As \glspl{esm} have evolved, they have become more sophisticated. Now,
they can model the behavior of energy systems in different social contexts,
such as the adoption of energy efficiency measures, the acceptance of energy
technologies, and the resistance to new energy projects. The Osier tool
\cite{Dotson_osier} is a framework for multi-objective optimization over traditional cost constraints that incorporates public preferences and user-defined parameters.

Pfenninger et al. \cite{pfenninger_energy_2014} describe four
paradigms of energy system modeling: optimization, simulation, econometric, and
hybrid models. In the optimization paradigm, the modeler seeks a normative
solution to a problem by minimizing or maximizing an objective function subject
to constraints. In the simulation paradigm, the modeler aims to predict the
behavior of the energy system by simulating the interactions between different
system components. In the econometric, or market, paradigm, the modeler seeks
to understand the relationship between different operational variables in the
energy system by estimating the parameters of a statistical model. The hybrid
paradigm is a catch-all for narrative scenarios that combine the paradigms to
develop a more comprehensive understanding of the energy system.

Although there are myriad paradigms of \gls{esm}, two philosophies (top-down
and bottom-up) to their construction dictate the restrictions a model will
place on the type of questions it can answer. In the top-down approach, the
modeler starts with a high-level view of the energy system and then drills into
the details. This approach aids in understanding the overall behavior of the
energy system and the impact of different policies on the system
\cite{laha_energy_2017}. In the bottom-up approach, the modeler starts with the
details of the energy system and then builds up to a high-level view. This
approach is useful for understanding the behavior of individual components of
the energy system and the impact of different technologies on the system
\cite{ipcc_ch2_2000,laha_energy_2017}.

