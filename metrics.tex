\section{Metrics}
\label{sec:metrics}

% discuss the metrics you are using
In this work, we will develop a model of nuclear energy in the \gls{usa} using concepts from \glspl{esm} on scenarios that compare the transition from our current fleet to incorporate advanced reactor technologies not currently deployed. To compare these scenarios, we have chosen to focus on a few key metrics: \gls{swu}, energy output, mass of fuel, and reactor deployment.

\subsection{Separative Work Units}
\label{sec:swu}
% justify what swu is and why it is a valuable metric
The process of enriching uranium is a critical step in the nuclear fuel cycle, and, as we have highlighted, is expected to be a bottle neck in the deployment of advanced reactors. \gls{swu}, or a Separative Work Unit, is a ubiquitous measure of effort that goes into producing nuclear fuel. It is simplified as:
\begin{align}
    SWU&= Q(C_p-C_f)
    \intertext{Where:}
    SWU&= \mbox{Separative Work Units [kgSWU]}\nonumber\\
    Q&= \mbox{ Quantity of material processed [kg]}\nonumber\\
    C_p&=\mbox{ Enrichment level of the product [$\%$]}\nonumber\\
    C_f&= \mbox{ Enrichment level of the feed [$\%$].}\nonumber
\end{align}

In this work, we will compare the \gls{swu} required for each scenario to understand the relative effort required to deploy the reactors and provide a stable precursor to economic calculations.

\subsection{Energy Output}
\label{sec:energy_output}

The deployment of reactors in this work is based on energy demand, which
approximates the complicated relationship that generators and utilities
have with power expansions. The reactors simulated herein have a static peak energy output, so the nuance in the fleet's ability to meet the demand comes from the deployment scheme and limitations in the fuel supply chain. We will devote time to discussing the realistic features of each scheme. \cyclus tracks the energy output of each reactor, and we will compare that with the demand scenarios to understand the relative performance of each deployment scheme.


\subsection{Mass of Fuel}
\label{sec:mass_of_fuel}

\cyclus tracks the mass of material in each transaction, in this work we will characterize the deployment challenge ahead of us using the fresh and used fuel accumulation to show 