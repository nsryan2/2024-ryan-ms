\section{Metrics}
\label{sec:metrics}

% discuss the metrics you are using
In this work, we will develop a model of nuclear energy in the \gls{usa} using concepts from \glspl{esm} on scenarios that compare the transition from our current fleet to incorporate advanced reactor technologies not currently deployed. To compare these scenarios, we have chosen to focus on a few key metrics: \gls{swu}, energy output, mass of fuel, and reactor deployment.

\subsection{Separative Work Units}
\label{sec:swu}

\gls{swu}, or Separative Work Units, is a ubiquitous measure of effort
that goes into producing nuclear fuel. It is simplified as:
\begin{align}
    SWU&= Q(C_p-C_f)
    \intertext{Where:}
    SWU&= \mbox{Separative Work Units [kgSWU]}\nonumber\\
    Q&= \mbox{ Quantity of material processed [kg]}\nonumber\\
    C_p&=\mbox{ Enrichment level of the product [$\%$]}\nonumber\\
    C_f&= \mbox{ Enrichment level of the feed [$\%$].}\nonumber
\end{align}

In this work, we will compare the \gls{swu} required for each scenario to understand the relative effort required to deploy the reactors and provide a stable precursor to economic calculations.

\subsection{Energy Output}
\label{sec:energy_output}

% just spit balling
The deployment of reactors in this work is based on energy demand, which
approximates the complicated relationship that generators and utilities
have with power expansions.

The reactors simulated herein have a static peak energy output, so the
nuance in the fleet's ability to meet the demand comes from the
deployment scheme and limitations in the fuel supply chain.

We have created a toy scenario to understand the ways in which the
deployment schemes under and over perform, and we will devote time to
discussing the realistic features of each scheme.

% sensitivity analysis?

\subsection{Mass of Fuel}
\label{sec:mass_of_fuel}

\cyclus has an understanding of the mass of material in each
transaction, in this work we will couple this tracking with an idea of
the volume of the fuel elements to get a relative sense of the volume
each scenario would produce.

We will further compare this with the storage capacity of proposed
projects in the \gls{usa}--like Yucca Mountain.
% improve with comprehensive list

In addition to the volume, the mass % sensitivity analysis?