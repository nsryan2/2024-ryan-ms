\begin{itemize}
    \item Greedy Algorithm: deploy the largest reactor first at each time step, fill in the remaining capacity with the next smallest, and so on.
    \item Pre-determined distributions: one or more reactors have a preset distribution, and a smaller capacity model fills in the gaps.
    \item Deployment Cap: there is a single-number capacity for one or more of the reactor models.
    \item Random Deployment: uses a date and hour as seed to randomly sample the reactors list.
    \item Initially Random, Greedy: randomly deploys reactors until a reactor bigger than the remaining capacity is proposed for each year, then fills remaining capacity with a greedy algorithm.
\end{itemize}

We apply each of these deployment schemes to a series of demand growth scenarios based on two predictions. The \gls{eia} publishes demand expansion projections for the totality of \gls{usa} .
% source name and citation with


\gls{doe}'s Liftoff Report for Nuclear ((((((NAME??? and SOURCE)))))) \cite{}
\begin{itemize}
    \item No growth [0$\%$]
    \item Low growth [1$\%$, 5$\%$, 10$\%$, 15$\%$]
    \item High growth [100$\%$, 200$\%$]
\end{itemize}
% sources for numbers and range

\subsection{Greedy Deployment}
% describe the algorithm in detail

% what are the realistic and the unrealistic parts

% when does it over or under perform


\subsection{Pre-determined Deployment}
% describe the algorithm in detail

% what are the realistic and the unrealistic parts

% when does it over or under perform


\subsection{Deployment Cap}
% describe the algorithm in detail

% what are the realistic and the unrealistic parts

% when does it over or under perform


\subsection{Random Deployment}
% describe the algorithm in detail

% what are the realistic and the unrealistic parts

% when does it over or under perform


\subsection{Initially Random, Greedy Deployment}
% describe the algorithm in detail

% what are the realistic and the unrealistic parts

% when does it over or under perform

